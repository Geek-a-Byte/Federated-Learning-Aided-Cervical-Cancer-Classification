% !TEX root = ../thesis.tex

\begin{abstract}
	
\parbox{\textwidth}{\centering\large\bf\thetitle}\\[10pt]
	
\noindent%
% Cervical cancer is one of the most common causes of mortality in women worldwide. There is a lack of effective screening programs in low-income countries for the detection and treatment of precancerous conditions. In the context of Bangladesh, cervical cancer is the second most common malignancy among women. Classification of pap-smear test cervical cell images is crucial as it gives essential information for the diagnosis of malignant or precancerous lesions and thus helps in providing a proper diagnosis. Most of the existing methods require accumulating pap-smear test images of all patients in a centralized location for classification purposes. However, this procedure hampers the privacy of patient data and creates data ownership issues. In this study, a novel convolutional neural network-based federated learning system is introduced for achieving both the objectives of accurate image classification and data privacy. Multiple hospitals across different countries can use the proposed system to train their local models with their private dataset without sharing it centrally, which eventually helps to build the central model of FL architecture with diverse datasets. Then the updates of the locally trained models get aggregated with an initially untrained global model in order to increase its performance. In traditional ML-based systems, the more the train data, the more efficiently the model performs. But, in this proposed system, clients can participate remotely to train a robust model even with the disadvantage of possessing a small dataset. In this study, a comparison was shown between traditional ML methods and the proposed CNN-based FL architecture where test precision, recall, f1-score and accuracy of 89.58\%, 88.47\%, 87.83\% and 88.46\% respectively was achieved for the global model which gave better results than that of traditional ML algorithms. This eventually proves the validation of the proposed method to be used for real-world environments.
Cervical cancer is one of the most common causes of mortality in women worldwide, and there is a lack of effective screening programs in low-income countries for the detection and treatment of precancerous conditions. Classification of pap-smear test cervical cell images is crucial as it gives essential information for the diagnosis of malignant or precancerous lesions and thus helps in providing a proper diagnosis. Most of the existing methods require accumulating pap-smear test images of all patients in a centralized location for classification purposes. However, this procedure may hamper the privacy of patient data and creates data ownership issues. In this study, a novel convolutional neural network based federated learning system is introduced for achieving both the objectives of accurate image classification and data privacy. In the proposed FL system, the updates of the locally trained models get aggregated with an initially untrained global model in order to increase its performance. In traditional ML-based systems, the more the train data, the more efficiently the model performs, but in the proposed system, clients can participate remotely to train a robust model even with the disadvantage of possessing a small dataset. The proposed CNN-based FL architecture showed the test precision, recall, f1-score and accuracy of 89.58\%, 88.47\%, 87.83\% and 88.46\% respectively. Thus multiple hospitals across different countries can use the proposed system to train their local models with their private dataset without sharing it centrally, which eventually helps to build the central model of FL architecture with diverse datasets.

\end{abstract}

% \centering
% left bottom right top
\begin{center}
      {\includegraphics[height=10.7in, width=180mm, trim={32mm 0 0.5mm 4cm}, clip]{chapters/abstractBN.pdf}} 
    \end{center}
% \includegraphics[height=10in, width = 165 mm]{chapters/abstractBN.pdf}
 

\noindent%


